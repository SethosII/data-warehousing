\section{Betriebswirtschaftliche Analyse}
\subsection{Beschreibung möglicher Anwendungen aus Business-Sicht}
Zur betriebswirtschaftlichen Analyse sollen die Daten der Datenbank aus Sicht des Kundenmanagements, der Umsatzentwicklung und -verteilung, der Artikel- und Kategoriepflege sowie des Bestell- und Lieferprozesses allgemein betrachtet werden. Für diese Perspektiven wurden folgende folgende Fragestellungen festgelegt:  
\begin{itemize}
  \item Kundenmanagement:
  \begin{itemize}
    \item Wie viele Kunden hat der Shop ? 
    \item Wie ist die Altersverteilung der Kunden ?
    \item Wie hoch ist das Durchschnittsalter ?
    \item Wie oft bestellt der selbe Kunde bzw. Was ist die typische Zeitspanne zwischen zwei Bestellungen ?
    \item Wie hoch ist das Verhältnis zwischen Reklamationen und Bestellungen ?
    \item Was sind die Hauptgründe für Reklamationen ?
    \item Wer sind meine Top-Kunden ?
    \item Was sind die profitabelsten Altersgruppen ?  
  \end{itemize}
  \item Umsatzverteilung und -entwicklung:
  \item Artikel- und Kategoriepflege:
  \item Bestell- und Lieferprozesse:
\end{itemize}


Zur konzeptuellen Modellierung des Data Warehouse sowie zur Beantwortung der gestellten Fragen, wurden die folgenden Kennzahlen für die einzelnen Perspektiven definiert:
\begin{itemize}
  \item 
\end{itemize}

\subsection{Konzeptuelle Modellierung}
\subsection{Datenverarbeitungsanforderungen}
