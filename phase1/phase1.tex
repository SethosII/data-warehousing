\section{Betriebswirtschaftliche Analyse}
\subsection{Beschreibung möglicher Anwendungen aus Business-Sicht}
Zur betriebswirtschaftlichen Analyse sollen die Daten der Datenbank aus Sicht des Kundenmanagements, der Umsatzentwicklung und -verteilung, der Artikel- und Kategoriepflege sowie des Bestell- und Lieferprozesses allgemein betrachtet werden. Für diese Perspektiven wurden folgende folgende Fragestellungen festgelegt:  
\begin{itemize}
  \item Kundenmanagement:
  	\begin{itemize}
    	\item Wie viele Kunden hat der Shop? 
    	\item Wie viele Neukunden hat der Shop?
    	\item Wie ist die Altersverteilung der Kunden?
    	\item Wie hoch ist das Durchschnittsalter?
   	 	\item Wie oft bestellt der selbe Kunde bzw. Was ist die typische Zeitspanne zwischen zwei Bestellungen?
    	\item Wie hoch ist das Verhältnis zwischen Reklamationen und Bestellungen?
    	\item Was sind die Hauptgründe für Reklamationen?
 
  	\end{itemize}
  \item Umsatzverteilung:
  	\begin{itemize}
		\item Wie hoch ist der Gesamtumsatz?    	
    	\item Wie ist die Verteilung nach Altersgruppen?
    	\item Was sind die umsatzstärksten Altersgruppen? 
  		\item Wie ist die Verteilung nach Produktgruppen?
  		\item Wie ist die Verteilung nach Postleitzahlen? 
  	\end{itemize}
  \item Artikel und Produktkategorien:
  	\begin{itemize}
  		\item Welche Produkte wurden am häufigsten gekauft ?
  		\item Welche Produkte wurden am häufigsten zurückgegeben?
  		\item Was waren die häufigsten Gründe für Retouren?
  	\end{itemize}
  \item Bestellungen:
  	\begin{itemize}
  		\item Wie viele Bestellungen wurden insgesamt aufgegeben?
  		\item Wie hoch ist der durchschnittliche Bestellwert?
  		\item Wie viele Bestellungen werden pro Altersgruppe aufgegeben?
  		\item Wie lang ist die durchschnittliche Dauer von vom Bestelldatum bis zum Zahlungseingang?
  	\end{itemize}	

    	

  	
  
  
  \item Bestell- und Lieferprozesse:
\end{itemize}


Zur konzeptuellen Modellierung des Data Warehouse sowie zur Beantwortung der gestellten Fragen, wurden die folgenden Kennzahlen für die einzelnen Perspektiven definiert:
\begin{itemize}
  \item 
\end{itemize}

\subsection{Konzeptuelle Modellierung}
\subsection{Datenverarbeitungsanforderungen}
